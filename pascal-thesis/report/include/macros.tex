%%% Local Variables:
%%% TeX-master: "../thesis"
%%% End:

%% macros
%%%%%%%%%
\newcommand{\loremipsum}{\todo[inline]{Lorem Ipsum}}

%% misc
%%%%%%%%%
\DeclareMathOperator{\pto}{\rightharpoonup}     % partial function arrow
\DeclareMathOperator{\powerset}{\mathcal{P}}    % powerset
\DeclareMathOperator{\proj}{\upharpoonright}    % projection
\DeclareMathOperator{\fst}{\pi_1}               % A x B → A
\DeclareMathOperator{\snd}{\pi_2}               % A x B → B
\DeclareMathOperator{\ft}{ft}                   % field domain assignment
\DeclareMathOperator{\rt}{rt}                   % register domain asignment
\newcommand{\p}[2]{#1(#2)}
\newcommand{\influence}[2]{#1 \triangleright #2}
\newcommand{\return}[1]{\triangledown #1}
\newcommand{\ins}[1]{\mathtt{#1}}
\newcommand{\fresh}[1]{\mathtt{fresh}(#1)}
\newcommand{\default}[1]{\mathtt{default}(#1)}
\DeclareMathOperator{\DVMseq}{DVM_{seq}}
\DeclareMathOperator{\DVMstat}{DVM_{stat}}
\DeclareMathOperator{\DVMdyn}{DVM_{dyn}}
\newcommand{\cs}[1]{\langle #1 \rangle}
\newcommand{\updatemds}{\mathit{update\text{-}mds}}

%% sets
%%%%%%%%%
\DeclareMathOperator{\nat}{\mathbb{N}}          % natural numbers
\DeclareMathOperator{\integer}{\mathbb{Z}}      % integers
\DeclareMathOperator{\C}{\mathcal{C}}           % classes
\DeclareMathOperator{\M}{\mathcal{M}}           % methods
\DeclareMathOperator{\Mid}{\mathcal{M}_{id}}    % method identifiers
\DeclareMathOperator{\F}{\mathcal{F}}           % fields
\DeclareMathOperator{\V}{\mathcal{V}}           % values
\DeclareMathOperator{\Val}{\mathbb{V}}          % instruction dependend values
\DeclareMathOperator{\Loc}{\mathcal{L}}         % locations
\DeclareMathOperator{\PP}{\mathcal{PP}}         % program points
\DeclareMathOperator{\R}{\mathcal{R}}           % registers
\DeclareMathOperator{\Heap}{\mathcal{H}}        % heaps
\DeclareMathOperator{\CS}{\mathcal{CS}}                   % call stacks
\DeclareMathOperator{\Conf}{\mathit{Conf}}      % configurations
\DeclareMathOperator{\LConf}{\mathit{Conf_l}}   % local configurations
\DeclareMathOperator{\Ins}{\mathcal{I}}         % instructions
\DeclareMathOperator{\InsSeq}{\mathcal{I}_{seq}}
\DeclareMathOperator{\InsStat}{\mathcal{I}_{stat}}
\DeclareMathOperator{\InsDyn}{\mathcal{I}_{dyn}}
\DeclareMathOperator{\Ann}{\mathit{Ann}}
\DeclareMathOperator{\Mod}{\mathit{Mod}}        % modes
\DeclareMathOperator{\Mds}{\mathit{Mds}}        % mode states
\DeclareMathOperator{\Conffinal}{\mathit{Conf}_F}               % final configurations
\DeclareMathOperator{\Events}{E}                % the set of events for thread creation and synchronization
\DeclareMathOperator{\Dom}{\mathcal{D}}      % the security domain
\DeclareMathOperator{\Obj}{\C \times \F \pto \V}                % objects

%% relations
%%%%%%%%%%%%%
\newcommand{\leadstolocal}[1]{\overset{#1}{\rightarrowtriangle}}        % local transition relation

%% elements
%%%%%%%%%%%%
\newcommand{\res}{v_{res}}             % return register
\DeclareMathOperator{\Prg}{P}                   % program
\DeclareMathOperator{\this}{\mathtt{this}}      % this pointer
\DeclareMathOperator{\asmnoread}{\mathit{asm-noread}}           % mode asm-noread
\DeclareMathOperator{\asmnowrite}{\mathit{asm-nowrite}}         % mode asm-nowrite
\DeclareMathOperator{\guarnoread}{\mathit{guar-noread}}         % mode guar-noread
\DeclareMathOperator{\guarnowrite}{\mathit{guar-nowrite}}       % mode guar-noread
\DeclareMathOperator{\confinitial}{\mathit{conf}_0}             % initial configuration
\DeclareMathOperator{\nullpointer}{\mathit{null}}                      % null pointer
\newcommand{\evententer}[1]{\blacklozenge #1}                   % event for entering a monitor region
\newcommand{\eventleave}[1]{\lozenge #1}         % event for leaving a monitor region
\DeclareMathOperator{\noevent}{\varepsilon}       % symbol for no event
\DeclareMathOperator{\emptycs}{\langle\rangle}
\newcommand{\low}{low}
\newcommand{\high}{high}
