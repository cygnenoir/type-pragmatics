\section{Motivation}
Smartphones have arrived in our daily lives. They help us organizing
and planing many aspects of our life, for example by reminding us of
our appointments, leading us the way to our leisure location and
helping us to stay in touch with our friends via social networks.

These features are provided by programs that can be installed on the
smartphone. To make use out of these programs we need to entrust them
with our private data. Moreover smartphones are by their nature
tightly connected to other devices and networks like the
internet. This raises the question whether a program ensures the
confidentiality of the entrusted data by \emph{not} leaking them to
untrusted parties.

Traditional security mechanisms like access control and encryption are
not enough to protect the confidentiality of the entrusted data,
because they only secure the access to the data (see
\cite{DBLP:journals/jsac/SabelfeldM03}). After a successful and
legitimate access, confidential data (or information about
confidential data) can be leaked -- accidentally or maliciously --
trough the data and control flow of the program, to publicly visible
outputs. An attacker can then try to conclude private information by
looking at the public outputs. The security mechanism that Android --
the most popular operating system for smartphones (see
\cite{gartner:android:marketshare}) -- provides to ensure the
confidentiality of the entrusted data is an access control mechanism
(see \cite{DBLP:conf/isw/DaviDSW10}). Thus the confidentiality of the
entrusted data is currently not ensured on Android if data is
accessed.

Leaks caused by the data and information flow can be described by
security properties that classify the data and information flows
within programs as legal and illegal. The absence of illegal data and
information flows is described with non-interference-like
\cite{DBLP:conf/sp/GoguenM82} properties, which require that public
outputs of a program are independent from private inputs. Due to this
independence an attacker is not able to conclude any private
information by looking at the public outputs.

Current security properties (e.g. \cite{alexandra:static:2011}) for
the \gls{dvm}, the software that executes programs on Android, are not
adequate for multithreaded programs, although most Android programs
use multiple threads (see \cite{Wognsen2012}). When constructing a
security property that is adequate for multithreaded programs a
desirable characteristic is compositionality, i.e. the parallel
execution of secure (w.r.t. the security property) threads is secure
again.

To achieve parallel compositionality many existing security properties
(e.g.\ \cite{DBLP:conf/csfw/SabelfeldS00} and
\cite{Zdancewic03observationaldeterminism}) make the worst case
assumption that the environment might access every variable at any
point in time. This over-approximation classifies intuitively secure
and useful programs as insecure. The security property introduced in
\cite{mantel.sands.ea:assumptions} is more precise, because it makes
the intended usage of variables explicit. Thus it does not need to
look at all variable accesses as it can be sure that some will not
happen. The variable usage pattern is made explicit by allowing each
thread to make assumptions on how variables are accessed by other
threads and by giving guarantees on how certain variables are access
by the thread itself. If all assumptions are matched by the
corresponding guarantees and the guarantees hold, it is permissible to
temporarily store confidential information in public variables and it
is possible to temporarily be sure that the values of certain
variables do not change during execution.

The purpose of this thesis is to adapt the SIFUM Security property
from \cite{mantel.sands.ea:assumptions} to an abstract transition
system that is capable of simulating the \gls{dvm} and to argue for
concrete instantiations of the transition system that the security
property is adequate.

%%% Local Variables:
%%% mode: latex
%%% TeX-master: "../thesis"
%%% End:
